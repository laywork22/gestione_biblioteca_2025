\documentclass[a4paper,12pt,oneside,openany]{report}
\usepackage[utf8]{inputenc}
\usepackage[T1]{fontenc}
\usepackage[italian]{babel}
\usepackage{longtable}
\usepackage{array}
\usepackage{multirow}
\usepackage{ragged2e}
\usepackage{float}
\usepackage{enumitem} % Per liste compatte

\renewcommand{\arraystretch}{1.3} 
\setlength{\tabcolsep}{8pt}

% Comando personalizzato per la scheda di dettaglio requisito
\newcommand{\requisito}[5]{
    \noindent\textbf{Nome requisito}: #1 \\
    \textbf{Prefisso identificativo}: #2 \\
    \textbf{Priorità}: #3 \\
    \textbf{Input}: #4 \\
    \textbf{Output}: #5 \par
    \vspace{0.5cm}
    \hrule
    \vspace{0.5cm}
}
\newcommand{\usecase}[7]{
    \noindent\textbf{Nome}: #1 \\
    \textbf{Dipendenze da requisiti}: #2 \\
    \textbf{Attori partecipanti}: #3 \\
    \textbf{Precondizioni}: #4 \\
    \textbf{Postcondizioni}: #5 \\
    \textbf{Flusso di eventi}: #6 \\
    \textbf{Flusso alternativo di eventi}: #7\par
    \vspace{0.5cm}
    \hrule
    \vspace{0.5cm}
}

\begin{document}
	
\begin{titlepage}
    \centering
    \vspace*{4cm}
    
    {\Huge \textbf{Documento di specifica dei requisiti}}\\
    \vspace{0.5cm}
    {\Huge \textbf{libraryManager 1.0}\par}
    
    \vspace{1.5cm}
    {\Large 1° Dicembre 2025\par}
    
    \vfill 
    
    \textbf{\textit{Gruppo 12}}\\
    \textit{Emiddio Ferrentino}\\
    \textit{Antonio D'Urso}\\
    \textit{Letizia Argenio}\\
    \textit{Rosa Genovese}\par
    
    \vspace{2cm} 
\end{titlepage}
	
\tableofcontents
	
\chapter*{Introduzione}
\addcontentsline{toc}{chapter}{Introduzione}
Il seguente documento descrive le specifiche dei requisiti di un sistema software per la gestione di una biblioteca universitaria. Il sistema, denominato \textit{libraryManager}, permette agli operatori di gestire il catalogo dei libri, l'anagrafica degli utenti e le operazioni di prestito e restituzione.

\chapter*{Categorizzazione dei requisiti}
\addcontentsline{toc}{chapter}{Categorizzazione dei requisiti}

\subsection*{Tabella sintetica dei requisiti}
\addcontentsline{toc}{section}{Tabella sintetica dei requisiti}

\setlength\LTleft{-1.1cm}
\begin{longtable}{|p{3cm}|p{2.2cm}|p{5cm}|p{2.26cm}|}
	\hline
	\textbf{AREA} & \textbf{PREFISSO} & \textbf{DESCRIZIONE} & \textbf{PRIORITÀ} \\
	\hline
	\endhead % Intestazione ripetuta su ogni pagina
	
	\multirow{10}{3cm}{Funzionalità \newline individuali\newline \textbf{IF}} 
	& IF-1.1 & 
	\textbf{Inserimento nuovo libro} \newline
    Il sistema deve permettere l'inserimento di un nuovo libro registrando titolo, autori, anno, ISBN e copie disponibili (vedi DF-1.1). & Alta \\
	\cline{2-4}
	& IF-1.2.1 & 
    \textbf{Modifica di un libro} \newline
    L'operatore deve poter modificare i metadati descrittivi di un libro esistente. & Alta \\
    \cline{2-4}
    &IF-1.2.2 & 
    \textbf{Aggiornamento copie} \newline
    L'operatore deve poter aggiornare il numero di copie disponibili. & Alta \\
	\cline{2-4}
	& IF-1.3 & 
	\textbf{Rimozione libro} \newline 
    Il sistema deve permettere di rimuovere un libro solo se non ci sono prestiti attivi per esso. & Alta \\
	\cline{2-4}
	& IF-1.4 & 
    \textbf{Visualizzazione libri} \newline
	Elenco libri ordinato di default per titolo, con ordinamento alternativo per anno o autore. & Media \\
	\cline{1-4}
	& IF-1.5 & 
    \textbf{Ricerca libro} \newline
	Ricerca per titolo, autore o ISBN. & Alta \\
	\cline{2-4}
	& IF-2.1 &
	\textbf{Registrazione utente} \newline 
    Inserimento nuovo utente con controllo unicità matricola (vedi DF-1.2). & Alta \\
	\cline{2-4}
	& IF-2.2 &
	\textbf{Modifica utente} \newline
    Modifica dati anagrafici utente esistente. & Alta \\
	\cline{2-4}
	& IF-2.3 &
	\textbf{Cancellazione utente} \newline
    Rimozione utente possibile solo se non ha prestiti in corso. & Media \\
	\cline{2-4}
	& IF-2.4 &
	\textbf{Visualizzazione utenti} \newline
    Lista utenti ordinabile per Cognome e Nome. & Alta \\
	\cline{2-4}
	&IF-2.5 &
	\textbf{Ricerca utente} \newline
     Ricerca per cognome o matricola. & Media \\
	\hline
	
	\multirow{5}{3cm}{Business flow\newline \textbf{BF}} 
	& BF-1.1 &
    \textbf{Vincolo max prestiti} \newline
	Il sistema blocca il prestito se l'utente ha già 3 prestiti attivi. & Alta \\
	\cline{2-4}
	& BF-1.2 &
    \textbf{Chiusura prestito} \newline
	Il sistema aggiorna lo stato a "restituito" e incrementa le copie disponibili. & Alta \\
	\cline{2-4}
	& BF-1.3 &
    \textbf{Visualizzazione prestiti} \newline
	Visualizzazione dello stato dei prestiti attivi (in regola/scaduti). & Media \\
    \cline{2-4}
    & BF-1.4 &
    \textbf{Vincolo rimozione libro} \newline
    Impossibile rimuovere un libro se è attualmente in prestito. & Alta \\
	\hline
    
	\multirow{5}{3cm}{Dati e formato\newline \textbf{DF}} 
	& DF-1.1 &
	\textbf{Dati Libro}: Titolo, Autori, Anno, ISBN, Copie totali, Copie disponibili. & Alta \\
    \cline{2-4}
    & DF-1.2 &
    \textbf{Dati Utente}: Nome, Cognome, Matricola, Email. & Alta \\
    \cline{2-4}
    & DF-1.3 &
    \textbf{Dati Prestito}: Utente, Libro, Data inizio, Data scadenza, Stato. & Alta \\
	\cline{2-4}
	& DF-1.4 &
    \textbf{Persistenza dati} \newline
    Tutte le informazioni devono essere salvate su file strutturato. & Alta \\
	\cline{2-4}
	& DF-1.5 &
	\textbf{Caricamento dati} \newline
    Caricamento automatico dei dati da file all'avvio. & Alta \\
	\hline
	
	\multirow{13}{3cm}{Interfaccia \newline Utente\newline \textbf{UI}} 
	& UI-1.1.1 &
    \textbf{Gestione libro} \newline
    Pulsanti espliciti per inserimento, modifica, rimozione e conferma. & Alta \\
    \cline{2-4}
    & UI-1.1.2 &
    \textbf{Lista libri} \newline
    Tabella libri ordinata per titolo. & Alta \\
    \cline{2-4}
    & UI-1.1.3 &
    \textbf{Ricerca libri} \newline
    Barra di ricerca per titolo, autore, ISBN. & Media \\
    \cline{2-4}
    & UI-1.1.4 &
    \textbf{Form libri} \newline
    Interfaccia per inserimento e modifica dati libri. & Media \\
    \cline{2-4}
    & UI-1.2.1 &
    \textbf{Gestione utente} \newline
    Interfaccia per inserimento e modifica dati utente. & Alta \\
    \cline{2-4}
    & UI-1.2.2 &
    \textbf{Lista utenti} \newline
    Tabella utenti ordinata per cognome/nome. & Alta \\
    \cline{1-4}
    & UI-1.2.3 &
    \textbf{Ricerca utenti} \newline
    Barra di ricerca per nome o cognome utente. & Media \\
    \cline{2-4}
    & UI-1.3.2 &
    \textbf{Gestione prestito} \newline
    Interfaccia per creare prestito e registrare restituzione. & Alta \\
    \cline{2-4}
    &  UI-1.3.1 &
    \textbf{Lista prestiti} \newline
    Tabella prestiti attivi con dettagli. & Alta \\
    \cline{2-4}
    & UI-2.1 &
    \textbf{Messaggi di errore} \newline
    Feedback per utente/libro non trovato o operazioni non consentite. & Media \\
    \cline{2-4}
    & UI-3.1 &
    \textbf{Prestiti scaduti} \newline
    Evidenziazione grafica dei prestiti in scadenza o ritardo. & Alta \\
    \hline
\end{longtable}

\subsection*{Schema di categorizzazione di dettaglio}
\addcontentsline{toc}{section}{Schema di categorizzazione di dettaglio}

\subsubsection*{FUNZIONALITÀ INDIVIDUALI (IF)}

\requisito
{Inserimento nuovo libro}
{IF-1.1}
{Alta}
{Titolo (stringa), Autori (lista stringhe), Anno (intero), ISBN (stringa univoca), Copie (intero)}
{Il sistema registra il nuovo libro in archivio e conferma l'operazione. Se l'ISBN esiste già, mostra un errore.}

\requisito
{Modifica metadati libro}
{IF-1.2.1}
{Alta}
{Selezione libro, nuovi valori per i campi modificabili}
{Il sistema aggiorna i dati del libro. L'aggiornamento è rifiutato se il nuovo ISBN entra in conflitto con un altro libro.}
\requisito
{Aggiornamento numero di copie}
{IF-1.2.2}
{Media}
{Selezione libro, numero di copie disponibili}
{Il sistema aggiorna il numero di copie disponibili di un libro a seguito dell'input dell'utente.
L'aggiornamento è rifiutato se il numero di copie disponibili è minore di zero.}
\requisito
{Rimozione libro}
{IF-1.3}
{Alta}
{Selezione libro da rimuovere}
{Se non ci sono prestiti attivi, il libro viene rimosso. Altrimenti, viene mostrato un messaggio di errore bloccante.}

\requisito
{Visualizzazione libri}
{IF-1.4}
{Media}
{Richiesta di visualizzazione libri}
{Il sistema mostra l'elenco dei libri ordinati per autore in ordine lessicografico, con la possibilità di ordinamento alternativo per anno e autore.
In caso di mancanza di libri viene mostrato un messaggio di avviso.}
\requisito
{Ricerca libro}
{IF-1.5}
{Alta}
{Parametro di ricerca,valore da cercare}
{Il sistema mostra una lista di libri ordinati lessicograficamente in base al parametro di ricerca.
La ricerca può essere effettuata in funzione del nome dell'autore, del titolo del libro o del codice ISBN.}

\requisito
{Registrazione nuovo utente}
{IF-2.1}
{Alta}
{Nome, Cognome, Matricola (univoca), Email}
{Il sistema salva il nuovo utente. Se la matricola è duplicata, l'inserimento viene respinto.}
\requisito
{Modifica dati utente}
{IF-2.2}
{Alta}
{Utente da modificare,Nome, Cognome, Matricola (univoca), Email}
{Il sistema salva il nuovo utente con i nuovi dati inseriti. Se la matricola è duplicata, l'inserimento viene respinto.}
\requisito
{Cancellazione utente}
{IF-2.3}
{Media}
{Utente da eliminare}
{Il sistema elimina l'utente selezionato. Se vengono rilevati prestiti attivi il processo di cancellazione viene respinto.}
\requisito
{Visualizzazione utenti}
{IF-2.4}
{Alta}
{Richiesta di visualizzazione lista utenti}
{Il sistema mostra l'elenco degli utenti ordinati per cognome in ordine lessicografico, con la possibilità di ordinamento alternativo per nome.
In caso di mancanza di utenti viene mostrato un messaggio di avviso.}
\requisito
{Ricerca utente}
{IF-2.5}
{Alta}
{Parametro di ricerca, valore da cercare}
{Il sistema mostra una lista di utenti ordinati lessicograficamente in base al parametro di ricerca coerenti con il valore da cercare.
La ricerca può essere effettuata in funzione del del cognome o della matricola.}

\newpage
\subsubsection*{BUSINESS FLOW (BF)}

\requisito
{Vincolo numero di prestiti}
{BF-1.1}
{Alta}
{Tentativo di creazione nuovo prestito per un utente}
{Il sistema verifica il numero di prestiti attivi dell'utente. Se sono già 3, blocca l'operazione e mostra un alert.}

\requisito
{Restituzione libro}
{BF-1.2}
{Alta}
{Selezione di un prestito attivo}
{Il sistema marca il prestito come "restituito", aggiorna la data di fine effettiva e incrementa di 1 le copie disponibili del libro.}
\requisito
{Visualizzazione prestiti}
{BF-1.3}
{Alta}
{Richiesta di visualizzazione prestiti attivi}
{Il sistema mostra lo stato dei prestiti attivi, evidenziando i prestiti scaduti o in fase di scadenza.}
\requisito
{Vincolo rimozione libro}
{BF-1.4}
{Alta}
{Richiesta rimozione di un libro attualmente in prestito}
{Il sistema non consente la rimozione di un libro qualora fosse attualmente occupato in un prestito attivo.}
\newpage
\subsubsection*{DATI E FORMATO (DF)}
\requisito
{Dati libro}
{DF-1.1}
{Alta}
{Formato dati del libro}
{Formattazione dati del libro in campi: Titolo, Autori, Anno, ISBN, Copie totali, Copie disponibili.}
\requisito
{Dati Utente}
{DF-1.2}
{Alta}
{Formato dati dell'utente}
{Formattazione dati dell'utente in campi: Nome, Cognome, Matricola, Email.}
\requisito
{Dati Prestito}
{DF-1.3}
{Alta}
{Formato dati del prestito attivo.}
{Formattazione dati del libro in campi: Utente, Libro, Data inizio, Data scadenza, Stato.}
\requisito
{Persistenza dati}
{DF-1.4}
{Alta}
{Comando di salvataggio o evento di chiusura applicazione}
{Tutti i dati (libri, utenti, storico prestiti) vengono serializzati su file (es. CSV o JSON) per garantire la persistenza tra le sessioni.}
\requisito
{Caricamento dati}
{DF-1.5}
{Alta}
{Avvio dell'applicazione libraryManager}
{Il sistema legge e deserializza i dati salvati su file (libri, utenti, prestiti) in memoria, rendendoli disponibili per l'uso immediato dell'operatore.}
\newpage
\subsubsection*{INTERFACCE UTENTE (UI)}

\requisito
{Controlli gestione libro}
{UI-1.1.1}
{Alta}
{Accesso dell'utente alla sezione di gestione libri}
{Pulsanti grafici espliciti per l'inserimento, la modifica e la conferma dei dati di un libro.}

\requisito
{Lista libri}
{UI-1.1.2}
{Alta}
{Accesso alla schermata di gestione catalogo}
{Visualizzazione tabellare di tutti i libri, ordinata per Titolo, con la possibilità di selezionare un libro per modificarlo o rimuoverlo.}

\requisito
{Ricerca libri}
{UI-1.1.3}
{Media}
{Accesso alla schermata di gestione catalogo}
{Barra di input per la ricerca testuale che filtra in tempo reale l'elenco dei libri in base a Titolo, Autore o ISBN.}

\requisito
{Form utente}
{UI-1.2.1}
{Alta}
{Richiesta di inserimento nuovo utente o selezione utente per la modifica}
{Interfaccia grafica con campi di input etichettati per Nome, Cognome, Matricola ed Email.}

\requisito
{Lista utenti}
{UI-1.2.2}
{Alta}
{Accesso alla schermata di gestione utenti}
{Visualizzazione tabellare di tutti gli utenti, ordinata per Cognome e Nome, con la possibilità di selezionare un utente per la modifica o la cancellazione.}

\requisito
{Lista prestiti}
{UI-1.3.1}
{Alta}
{Accesso alla schermata di gestione prestiti}
{Tabella contenente i dettagli dei prestiti attivi (Utente, Libro, Data Inizio, Data Scadenza, Stato).}

\requisito
{Gestione prestito}
{UI-1.3.2}
{Alta}
{Selezione di un utente e di un libro per creare un prestito, o selezione di un prestito attivo per la restituzione}
{Interfaccia per la creazione di un nuovo prestito e pulsante dedicato per la registrazione della restituzione del libro selezionato.}

\requisito
{Messaggi di errore}
{UI-2.1}
{Media}
{Tentativo di operazione non valida (es. ISBN duplicato, utente con 3 prestiti, libro in prestito)}
{Visualizzazione di un messaggio di feedback chiaro e descrittivo per l'operatore, che indichi la ragione dell'errore o dell'operazione non consentita.}

\requisito
{Prestiti scaduti}
{UI-3.1}
{Alta}
{Visualizzazione della Lista prestiti (UI-1.3.1)}
{Evidenziazione grafica (es. colore rosso) applicata a tutta la riga dei prestiti la cui data di scadenza è stata superata o che sono in prossimità della scadenza (es. meno di 3 giorni rimanenti).}
\chapter*{Casi d'uso}
\addcontentsline{toc}{chapter}{Casi d'uso}

\section*{Descrizione dei casi d'uso}
\textit{placeholder}

\begin{enumerate}
    \item ciao
    \item no
    
    
\end{enumerate}

\section*{Diagramma dei casi d'uso}
\textit{placeholder}

\end{document}
