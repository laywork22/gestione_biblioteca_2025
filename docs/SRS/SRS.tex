\documentclass[a4paper,12pt,oneside,openany]{report}
\usepackage[utf8]{inputenc}
\usepackage[T1]{fontenc}
\usepackage[italian]{babel}
\usepackage{longtable}
\usepackage{array}
\usepackage{multirow}
\usepackage{ragged2e}
\usepackage{float}
\usepackage{enumitem} % Per liste personalizzate
\usepackage{graphicx}
\usepackage[hidelinks]{hyperref}
\usepackage{pdfpages}

% --- DEFINIZIONE LISTE ---
\setlist[enumerate]{nosep} % Rimuove spazio tra gli item nelle liste standard

% Nuova lista per i Flussi Alternativi (es. 4a.1)
\newlist{altflow}{enumerate}{2} 
% Livello 1 (es. 4a.): Stampa numero+a.
\setlist[altflow,1]{label=\arabic*a., ref=\arabic*a, nosep}
% Livello 2 (es. 4a.1.): Stampa livello1+punto+numero
\setlist[altflow,2]{label=\thealtflowi.\arabic*., nosep}


\renewcommand{\arraystretch}{1.3} 
\setlength{\tabcolsep}{8pt}

% Comando personalizzato per la scheda di dettaglio requisito
\newcommand{\requisito}[5]{
	\noindent\textbf{Nome requisito}: #1 \\
	\textbf{Prefisso identificativo}: #2 \\
	\textbf{Priorità}: #3 \\
	\textbf{Input}: #4 \\
	\textbf{Output}: #5 \par
	\vspace{0.5cm}
	\hrule
	\vspace{0.5cm}
}
\newcommand{\requisitonohrule}[5]{
	\noindent\textbf{Nome requisito}: #1 \\
	\textbf{Prefisso identificativo}: #2 \\
	\textbf{Priorità}: #3 \\
	\textbf{Input}: #4 \\
	\textbf{Output}: #5 \par
	\vspace{1cm}
}

\newcommand{\usecase}[7]{
	\noindent\textbf{Nome}: #1 \\
	\textbf{Dipendenze da requisiti}: #2 \\
	\textbf{Attori partecipanti}: #3 \\
	\textbf{Precondizioni}: #4 \\
	\textbf{Postcondizioni}: #5 \par
	\vspace{0.5cm} 
	\noindent\textbf{Flusso di eventi}:\\
	#6 \par
	\vspace{0.5cm} 
	\noindent\textbf{Flusso alternativo di eventi}:\\
	#7\par
	\vspace{0.5cm}
	\hrule
	\vspace{0.5cm}
}

\newcommand{\usecasenohrule}[7]{
	\noindent\textbf{Nome}: #1 \\
	\textbf{Dipendenze da requisiti}: #2 \\
	\textbf{Attori partecipanti}: #3 \\
	\textbf{Precondizioni}: #4 \\
	\textbf{Postcondizioni}: #5 \par
	\vspace{0.5cm} 
	\noindent\textbf{Flusso di eventi}:\\
	#6 \par
	\vspace{0.5cm} 
	\noindent\textbf{Flusso alternativo di eventi}:\\
	#7\par
	\vspace{1cm}
}

\begin{document}
	
	\begin{titlepage}
		\centering
		\vspace*{4cm}
		
		{\Huge \textbf{Documento di specifica dei requisiti}}\\
		\vspace{0.5cm}
		{\Huge \textbf{libraryManager 1.0}\par}
		
		\vspace{1.5cm}
		{\Large 1° Dicembre 2025\par}
		
		\vfill 
		
		\textbf{\textit{Gruppo 12}}\\
		\textit{Emiddio Ferrentino}\\
		\textit{Antonio D'Urso}\\
		\textit{Letizia Argenio}\\
		\textit{Rosa Genovese}\par
		
		\vspace{2cm} 
	\end{titlepage}
	
	\tableofcontents
	
	\chapter*{Introduzione}
	\addcontentsline{toc}{chapter}{Introduzione}
	Il seguente documento descrive le specifiche dei requisiti di un sistema software per la gestione di una biblioteca universitaria. Il sistema, denominato \textit{libraryManager}, permette agli operatori di gestire il catalogo dei libri, l'anagrafica degli utenti e le operazioni di prestito e restituzione.
	
	\chapter*{Categorizzazione dei requisiti}
	\addcontentsline{toc}{chapter}{Categorizzazione dei requisiti}
	
	\subsection*{Tabella sintetica dei requisiti}
	
	\addcontentsline{toc}{section}{Tabella sintetica dei requisiti}
	
	\setlength\LTleft{-1.1cm}
	\begin{longtable}{|p{3cm}|p{2.2cm}|p{5cm}|p{2.26cm}|}
		\hline
		\textbf{AREA} & \textbf{PREFISSO} & \textbf{DESCRIZIONE} & \textbf{PRIORITÀ} \\
		\hline
		\endhead % Intestazione ripetuta su ogni pagina
		
		\multirow{10}{3cm}{Funzionalità \newline individuali\newline \textbf{IF}} 
		& IF-1.1 & 
		\textbf{Inserimento nuovo libro} \newline
		Il sistema deve permettere l'inserimento di un nuovo libro registrando titolo, autori, anno, ISBN e copie disponibili (vedi DF-1.1). & Alta \\
		\cline{2-4}
		& IF-1.2.1 & 
		\textbf{Modifica metadati libro} \newline
		L'operatore deve poter modificare i metadati descrittivi di un libro esistente. & Alta \\
		\cline{2-4}
		&IF-1.2.2 & 
		\textbf{Aggiornamento numero di copie} \newline
		L'operatore deve poter aggiornare il numero di copie disponibili. & Alta \\
		\cline{2-4}
		& IF-1.3 & 
		\textbf{Cancellazione libro} \newline 
		Il sistema deve permettere di rimuovere un libro solo se non ci sono prestiti attivi per esso. & Alta \\
		\cline{2-4}
		& IF-1.4 & 
		\textbf{Visualizzazione libri} \newline
		Elenco libri ordinato di default per titolo, con ordinamento alternativo per anno o autore. & Media \\
		\cline{1-4}
		& IF-1.5 & 
		\textbf{Ricerca libro} \newline
		Ricerca per titolo, autore o ISBN. & Alta \\
		\cline{2-4}
		& IF-2.1 &
		\textbf{Registrazione utente} \newline 
		Inserimento nuovo utente con controllo unicità matricola (vedi DF-1.2). & Alta \\
		\cline{2-4}
		& IF-2.2 &
		\textbf{Modifica utente} \newline
		Modifica dati anagrafici utente esistente. & Alta \\
		\cline{2-4}
		& IF-2.3 &
		\textbf{Cancellazione utente} \newline
		Cancellazione utente possibile solo se non ha prestiti in corso. & Media \\
		\cline{2-4}
		& IF-2.4 &
		\textbf{Visualizzazione utenti} \newline
		Lista utenti ordinabile per Cognome e Nome. & Alta \\
		\cline{2-4}
		&IF-2.5 &
		\textbf{Ricerca utente} \newline
		Ricerca per cognome o matricola. & Media \\
		\cline{2-4}
		&IF-2.6 &
		\textbf{Registrazione prestito} \newline
		Inserimento nuovo prestito con controllo vincolo sul numero di prestiti e inserimento data restituzione arbitraria eseguita dall'operatore. & Alta \\
		\cline{2-4}
        & IF-2.7 &
        \textbf{Ricerca prestito attivo}\newline Ricerca del prestito in base a utente o libro &
        Media\\
        \cline{2-4}
     & IF-2.8 &
    \textbf{Cancellazione prestito } \newline
    Cancellazione di un prestito a stato attivo inserito.
    &
Alta \\

\hline
        \newpage
		\multirow{5}{3cm}{Business flow\newline \textbf{BF}} 
		& BF-1.1 &
		\textbf{Vincolo max prestiti} \newline
		Il sistema blocca il prestito se l'utente ha già 3 prestiti attivi. & Alta \\
		\cline{2-4}
		& BF-1.2 &
		\textbf{Chiusura prestito} \newline
		Il sistema aggiorna lo stato a "restituito" e incrementa le copie disponibili. & Alta \\
		\cline{2-4}
		& BF-1.3 &
		\textbf{Visualizzazione prestiti} \newline
		Visualizzazione dello stato dei prestiti attivi (in regola/scaduti) ordinati per data di scadenza. & Media \\
		\cline{2-4}
		& BF-1.4 &
		\textbf{Vincolo cancellazione libro} \newline
 		Impossibile rimuovere un libro se è attualmente in prestito. & Alta \\
		\hline
		\space
		\multirow{5}{3cm}{Dati e formato\newline \textbf{DF}} 
		& DF-1.1 &
		\textbf{Dati Libro}: Titolo, Autori, Anno, ISBN, Copie totali, Copie disponibili. & Alta \\
		\cline{2-4}
		& DF-1.2 &
		\textbf{Dati Utente}: Nome, Cognome, Matricola, Email. & Alta \\
		\cline{2-4}
		& DF-1.3 &
		\textbf{Dati Prestito}: Utente, Libro, Data inizio, Data scadenza, Stato. & Alta \\
		\cline{2-4}
		& DF-1.4 &
		\textbf{Persistenza dati} \newline
		Tutte le informazioni devono essere salvate su file strutturato. & Alta \\
		\cline{2-4}
		& DF-1.5 &
		\textbf{Caricamento dati} \newline
		Caricamento automatico dei dati da file all'avvio. & Alta \\
		\hline
		\newpage
		\multirow{13}{3cm}{Interfaccia \newline Utente\newline \textbf{UI}} 
		& UI-1.1.1 &
		\textbf{Gestione libro} \newline
		Pulsanti espliciti per inserimento, modifica, cancellazione e conferma. & Alta \\
		\cline{2-4}
		& UI-1.1.2 &
		\textbf{Lista libri} \newline
		Tabella libri ordinata per titolo. & Alta \\
		\cline{2-4}
		& UI-1.1.3 &
		\textbf{Ricerca libri} \newline
		Barra di ricerca per titolo, autore, ISBN. & Media \\
		\cline{2-4}
	
		& UI-1.2.1 &
		\textbf{Gestione utente} \newline
		Interfaccia per inserimento e modifica dati utente. & Alta \\
		\cline{2-4}
		& UI-1.2.2 &
		\textbf{Lista utenti} \newline
		Tabella utenti ordinata per cognome/nome. & Alta \\
		\cline{2-4}
		& UI-1.2.3 &
		\textbf{Ricerca utenti} \newline
		Barra di ricerca per nome o cognome utente. & Media \\
		\cline{2-4}
		& UI-1.3.2 &
		\textbf{Gestione prestito} \newline
		Interfaccia per creare prestito e registrare restituzione. & Alta \\
		\cline{2-4}
		&  UI-1.3.1 &
		\textbf{Lista prestiti} \newline
		Tabella prestiti attivi con dettagli ordinati per data di scadenza. & Alta \\
		\cline{2-4}
		& UI-2.1 &
		\textbf{Messaggi di errore} \newline
		Feedback per utente/libro non trovato o operazioni non consentite. & Media \\
		\cline{2-4}
		& UI-3.1 &
		\textbf{Prestiti scaduti} \newline
		Evidenziazione grafica dei prestiti in scadenza o ritardo. & Alta \\
		\hline
	\end{longtable}
	\newpage
	\subsection*{Schema di categorizzazione di dettaglio}
	\addcontentsline{toc}{section}{Schema di categorizzazione di dettaglio}
	
	\subsubsection*{FUNZIONALITÀ INDIVIDUALI (IF)}
	\addcontentsline{toc}{subsection}{Funzionalità individuali}
	\requisito
	{Inserimento nuovo libro}
	{IF-1.1}
	{Alta}
	{Titolo (stringa), Autori (lista stringhe), Anno (intero), ISBN (stringa univoca), Copie (intero)}
	{Il sistema registra il nuovo libro in archivio e conferma l'operazione. Se l'ISBN esiste già, sarà incrementato il contatore di copie disponibili di quel libro di uno.}
	
	\requisito
	{Modifica metadati libro}
	{IF-1.2.1}
	{Alta}
	{Selezione libro, nuovi valori per i campi modificabili}
	{Il sistema aggiorna i dati del libro. L'aggiornamento è rifiutato se il nuovo ISBN entra in conflitto con un altro libro.}
	\requisito
	{Aggiornamento numero di copie}
	{IF-1.2.2}
	{Alta}
	{Selezione libro, numero di copie disponibili}
	{Il sistema aggiorna il numero di copie disponibili di un libro a seguito dell'input dell'utente.
		L'aggiornamento è rifiutato se il numero di copie disponibili è negativo.}
	
	
	\requisito
	{Cancellazione libro}
	{IF-1.3}
	{Alta}
	{Selezione libro da rimuovere}
	{Se non ci sono prestiti attivi, il libro viene rimosso. Altrimenti, viene mostrato un messaggio di errore e l'operazione viene annullata.}
	
	\requisito
	{Visualizzazione libri}
	{IF-1.4}
	{Media}
	{Richiesta di visualizzazione libri}
	{Il sistema mostra l'elenco dei libri ordinati per titolo in ordine lessicografico, con la possibilità di ordinamento alternativo per autore.
		Se l'archivio è vuoto viene mostrato un messaggio di avviso.}
	\requisito
	{Ricerca libro}
	{IF-1.5}
	{Alta}
	{Criterio di ricerca(es. Titolo),valore da cercare}
	{Il sistema mostra una lista di libri ordinati lessicograficamente in base al parametro di ricerca.
		La ricerca può essere effettuata in funzione del nome dell'autore, del titolo del libro o del codice ISBN.}
	
	\requisito
	{Registrazione nuovo utente}
	{IF-2.1}
	{Alta}
	{Nome, Cognome, Matricola (univoca), Email}
	{Il sistema salva il nuovo utente. Se la matricola è duplicata, l'inserimento viene respinto.}
	\requisito
	{Modifica dati utente}
	{IF-2.2}
	{Alta}
	{Utente da modificare,Nome, Cognome, Matricola (univoca), Email}
	{Il sistema salva il nuovo utente con i nuovi dati inseriti. Se la matricola è duplicata, l'inserimento viene respinto.}
	\requisito
	{Cancellazione utente}
	{IF-2.3}
	{Media}
	{Utente da eliminare}
	{Il sistema elimina l'utente selezionato. Se vengono rilevati prestiti attivi il processo di cancellazione viene respinto.}
	\requisito
	{Visualizzazione utenti}
	{IF-2.4}
	{Alta}
	{Richiesta di visualizzazione lista utenti}
	{Il sistema mostra l'elenco degli utenti ordinati per cognome in ordine lessicografico, con la possibilità di ordinamento alternativo per nome.
		In caso di mancanza di utenti viene mostrato un messaggio di avviso.}
	\requisito
	{Ricerca utente}
	{IF-2.5}
	{Alta}
	{Parametro di ricerca, valore da cercare}
	{Il sistema mostra una lista di utenti ordinati lessicograficamente in base al parametro di ricerca coerenti con il valore da cercare.
		La ricerca può essere effettuata in funzione del del cognome o della matricola.}
\requisito
    {Registrazione prestito}
    {IF-2.6}
    {Alta}
    {Utente richiedente, Libro da prestare, Data di restituzione}
    {Il sistema crea un nuovo record di prestito associando l'utente al libro con la data di restituzione indicata dall'operatore. L'operazione viene bloccata se l'utente ha raggiunto il limite massimo di prestiti o se le copie disponibili del libro sono esaurite.}

\requisito
    {Ricerca prestito}
    {IF-2.7}
    {Media}
    {Parametro di ricerca (Matricola utente, Cognome o Titolo libro), Valore}
    {Il sistema filtra l'elenco dei prestiti attivi mostrando solo quelli corrispondenti ai criteri inseriti. Permette di individuare rapidamente prestiti specifici per verificarne lo stato o registrarne la restituzione.}
    \requisito
    {Cancellazione prestito}
    {IF-2.8}
    {Alta}
    {Selezione del prestito da eliminare}
    {Il sistema rimuove il record del prestito e incrementa le copie disponibili del libro.}
	\newpage
	\subsubsection*{BUSINESS FLOW (BF)}
    \addcontentsline{toc}{subsection}{Business Flow}

	\requisito
	{Vincolo numero di prestiti}
	{BF-1.1}
	{Alta}
	{Tentativo di creazione nuovo prestito per un utente}
	{Il sistema verifica il numero di prestiti attivi dell'utente. Se sono già 3, blocca l'operazione e mostra un alert.}
	
	\requisito
	{Restituzione libro}
	{BF-1.2}
	{Alta}
	{Selezione di un prestito attivo}
	{Il sistema marca il prestito come "restituito", aggiorna la data di fine effettiva e incrementa di 1 le copie disponibili del libro.}
	\requisito
	{Visualizzazione prestiti}
	{BF-1.3}
	{Alta}
	{Richiesta di visualizzazione prestiti attivi}
	{Il sistema mostra lo stato dei prestiti attivi ordinati per la data di scadenza, evidenziando i prestiti scaduti o prossimi alla scadenza.}
	\requisito
	{Vincolo cancellazione libro}
	{BF-1.4}
	{Alta}
	{Richiesta cancellazione di un libro attualmente in prestito}
	{Il sistema non consente la cancellazione di un libro qualora fosse attualmente associato a un prestito attivo.}
	\newpage
	\subsubsection*{DATI E FORMATO (DF)}
    \addcontentsline{toc}{subsection}{Dati e formato dei dati}

	\requisito
	{Dati libro}
	{DF-1.1}
	{Alta}
	{Formato dati del libro}
	{Formattazione dati del libro in campi: Titolo, Autori, Anno, ISBN, Copie totali, Copie disponibili.}
	\requisito
	{Dati Utente}
	{DF-1.2}
	{Alta}
	{Formato dati dell'utente}
	{Formattazione dati dell'utente in campi: Nome, Cognome, Matricola, Email. La email registrata deve essere istituzionale, nel formato *@studenti.unisa.it}
	\requisito
	{Dati Prestito}
	{DF-1.3}
	{Alta}
	{Formato dati del prestito attivo.}
	{Formattazione dati del libro in campi: Utente, Libro, Data inizio, Data scadenza, Stato.}
	\requisito
	{Persistenza dati}
	{DF-1.4}
	{Alta}
	{Comando di salvataggio o evento di chiusura applicazione}
	{Tutti i dati (libri, utenti, storico prestiti) vengono serializzati su file (es. CSV o JSON) per garantire la persistenza tra le sessioni.}
	\requisito
	{Caricamento dati}
	{DF-1.5}
	{Alta}
	{Avvio dell'applicazione libraryManager}
	{Il sistema legge e deserializza i dati salvati su file (libri, utenti, prestiti) in memoria, rendendoli disponibili per l'uso immediato dell'operatore.}
	\newpage
	\subsubsection*{INTERFACCIA UTENTE (UI)}
    \addcontentsline{toc}{subsection}{Interfaccia Utente}

	\requisito
	{Controlli gestione libro}
	{UI-1.1.1}
	{Alta}
	{Accesso dell'utente alla sezione di gestione libri}
	{Pulsanti grafici espliciti per l'inserimento, la modifica e la conferma dei dati di un libro.}
	
	\requisito
	{Lista libri}
	{UI-1.1.2}
	{Alta}
	{Accesso alla schermata di gestione catalogo}
	{Visualizzazione tabellare di tutti i libri, ordinata per Titolo, con la possibilità di selezionare un libro per modificarlo o rimuoverlo.}
	
	\requisito
	{Ricerca libri}
	{UI-1.1.3}
	{Media}
	{Accesso alla schermata di gestione catalogo}
	{Barra di input per la ricerca testuale che filtra in tempo reale l'elenco dei libri in base a Titolo, Autore o ISBN.}
	
	\requisito
	{Form utente}
	{UI-1.2.1}
	{Alta}
	{Richiesta di inserimento nuovo utente o selezione utente per la modifica}
	{Interfaccia grafica con campi di input etichettati per Nome, Cognome, Matricola ed Email.}
	
	\requisito
	{Lista utenti}
	{UI-1.2.2}
	{Alta}
	{Accesso alla schermata di gestione utenti}
	{Visualizzazione tabellare di tutti gli utenti, ordinata per Cognome e Nome, con la possibilità di selezionare un utente per la modifica o la cancellazione.}
	
	\requisito
	{Ricerca utenti}
	{UI-1.2.3}
	{Media}
	{Accesso alla lista utenti}
	{Barra di ricerca che permette di filtrare la lista degli utenti visualizzati inserendo parzialmente il cognome o la matricola.}
	
	\requisito
	{Lista prestiti}
	{UI-1.3.1}
	{Alta}
	{Accesso alla schermata di gestione prestiti}
	{Tabella contenente i dettagli dei prestiti attivi (Utente, Libro, Data Inizio, Data Scadenza, Stato) ordinata per data di scadenza.}
	
	\requisito
	{Gestione prestito}
	{UI-1.3.2}
	{Alta}
	{Selezione di un utente e di un libro per creare un prestito, o selezione di un prestito attivo per la restituzione}
	{Interfaccia per la creazione di un nuovo prestito e pulsante dedicato per la registrazione della restituzione del libro selezionato.}
	
	\requisito
	{Messaggi di errore}
	{UI-2.1}
	{Media}
	{Tentativo di operazione non valida (es. ISBN duplicato, utente con 3 prestiti, libro in prestito)}
	{Visualizzazione di un messaggio di feedback chiaro e descrittivo per l'operatore, che indichi la ragione dell'errore o dell'operazione non consentita.}
	
\noindent\textbf{Nome requisito}: Prestiti scaduti \\
	\textbf{Prefisso identificativo}: UI-3.1 \\
	\textbf{Priorità}: Alta \\
	\textbf{Input}: Visualizzazione della Lista prestiti (UI-1.3.1) \\
	\textbf{Output}: Tutta la riga dei prestiti la cui data di scadenza è stata superata o è prossima alla fine sarà evidenziata. \par
	
	\vspace{0.5cm}

    
    \chapter*{Casi d'uso}
	\addcontentsline{toc}{chapter}{Casi d'uso}
	
	\subsection*{Descrizione dei casi d'uso}
	\addcontentsline{toc}{section}{Descrizione dei casi d'uso}
	
	\usecasenohrule
	{Inserimento nuovo libro}
	{IF-1.1, IF-1.2.2, DF-1.1, UI-1.1.1}
	{Bibliotecario}
	{Il libro da inserire non è presente nell'elenco dei libri.}
	{Nell’elenco dei libri è stato inserito un nuovo record e il numero copie è aggiornato.}
	{
		\begin{enumerate}
			\item Il bibliotecario accede alla schermata "Area Libri".
			\item Seleziona la funzione "Aggiungi libro".
			\item Inserisce i dati: Titolo, Autori, Anno, ISBN, Numero copie.
			\item Il sistema controlla tramite l’ISBN che non sia già presente un’altra copia dello stesso libro.
			\item Il bibliotecario conferma l’operazione.
			\item Il sistema aggiorna l’elenco dei libri e il numero di copie del libro in questione.
			\item Il sistema visualizza un messaggio di conferma operazione.
		\end{enumerate}
	}
	{
		\begin{altflow}[start=4]
			\item {Libro già registrato (ISBN duplicato)}
			\begin{altflow}
				\item Il sistema rileva che l'ISBN esiste già.
				\item Il sistema chiede al bibliotecario se intende aggiornare il numero di copie del libro esistente.
				\item Se affermativo, il flusso prosegue aggiornando l'inventario (vedi IF-1.2.2).
			\end{altflow}
		\end{altflow}
		
		\begin{altflow}[start=5]
			\item {Annullamento}
			\begin{altflow}
				\item Il bibliotecario annulla l’operazione e torna alla lista libri.
			\end{altflow}
		\end{altflow}
	}
	
	\usecase
	{Modifica metadati libro}
	{IF-1.2.1,UI-1.1.1}
	{Bibliotecario}
	{Il libro deve essere già registrato a sistema.}
	{I metadati sono aggiornati e persistiti nel database.}
	{
		\begin{enumerate}
			\item Il bibliotecario accede alla schermata "Area Libri".
			\item Seleziona un libro dall’elenco per visualizzarne i dettagli.
			\item Seleziona la funzione "Modifica dati".
			\item Modifica i dati editabili: Titolo, Autori, Anno o Numero copie (ISBN non modificabile).
			\item Il bibliotecario conferma l’operazione.
			\item Il sistema salva le modifiche.
		\end{enumerate}
	}
	{
		\begin{altflow}[start=6]
			\item {Annullamento}
			\begin{altflow}
				\item Il bibliotecario annulla l’operazione; i dati restano invariati.
			\end{altflow}
		\end{altflow}
	}
	
	\usecase
	{Ricerca libro}
	{IF-1.5, UI-1.1.3}
	{Bibliotecario}
	{L'archivio libri non è vuoto.}
	{Visualizzazione dell'elenco libri filtrato secondo i criteri immessi.}
	{
		\begin{enumerate}
			\item Il bibliotecario accede alla schermata "Area Libri".
			\item Accede alla barra o sezione di ricerca.
			\item Inserisce il criterio di ricerca: parte del Titolo, Autore o ISBN esatto.
			\item Seleziona il comando "Cerca".
			\item Il sistema filtra la lista mostrando solo i libri corrispondenti ai criteri.
			\item Il bibliotecario visualizza i risultati.
		\end{enumerate}
	}
	{
		\begin{altflow}[start=5]
			\item {Nessun risultato}
			\begin{altflow}
				\item Il sistema non trova corrispondenze.
				\item Viene mostrato il messaggio "Nessun libro trovato".
			\end{altflow}
		\end{altflow}
	}
	
	\usecase
	{Visualizzazione catalogo libri}
	{IF-1.4, UI-1.1.2}
	{Bibliotecario}
	{Almeno un libro è presente nel sistema.}
	{Il bibliotecario visualizza l'elenco completo ordinato.}
	{
		\begin{enumerate}
			\item Il bibliotecario accede alla schermata "Area Libri".
			\item Il sistema recupera la lista dei libri.
			\item Il sistema mostra di default l'elenco ordinato alfabeticamente per Titolo.
		\end{enumerate}
	}
	{
		\begin{altflow}[start=3]
			\item {Cambio ordinamento}
			\begin{altflow}
				\item Il bibliotecario seleziona l'ordinamento per "Anno" o "Autore".
				\item Il sistema riordina la lista secondo il criterio scelto.
			\end{altflow}
		\end{altflow}
	}
	
	\usecase
	{Cancellazione libro}
	{IF-1.3, BF-1.4}
	{Bibliotecario}
	{Il libro da rimuovere è selezionato.}
	{Il libro viene rimosso logicamente o fisicamente dall'archivio.}
	{
		\begin{enumerate}
			\item Il bibliotecario seleziona il libro da eliminare dalla lista.
			\item Seleziona il comando "Rimuovi libro".
			\item Il sistema verifica che non ci siano prestiti attivi associati a quel libro (Vincolo BF-1.4).
			\item Il sistema chiede conferma dell'operazione.
			\item Il bibliotecario conferma.
			\item Il sistema rimuove il libro e aggiorna l'elenco.
		\end{enumerate}
	}
	{
		\begin{altflow}[start=3]
			\item {Violazione vincolo prestiti}
			\begin{altflow}
				\item Il sistema rileva prestiti attivi per il libro.
				\item Il sistema blocca la cancellazione e mostra l'errore: "Impossibile rimuovere: copie attualmente in prestito".
			\end{altflow}
		\end{altflow}
	}
	
	\usecasenohrule
	{Registrazione utente}
	{IF-2.1, DF-1.2}
	{Bibliotecario}
	{L’utente non è già presente a sistema.}
	{Nuovo utente registrato e abilitato ai servizi.}
	{
		\begin{enumerate}
			\item Il bibliotecario accede alla schermata "Area Utenti".
			\item Seleziona "Aggiungi utente".
			\item Inserisce i dati: Nome, Cognome, Matricola, Email.
			\item Il sistema verifica che la Matricola non sia già associata ad un altro utente.
			\item Il bibliotecario conferma l’operazione.
			\item Il sistema salva il nuovo utente e aggiorna la lista.
			\item Viene visualizzato un messaggio di successo.
		\end{enumerate}
	}
	{
		\begin{altflow}[start=4]
			\item {Matricola duplicata}
			\begin{altflow}
				\item Il sistema rileva una matricola esistente.
				\item Mostra l'errore "Utente già registrato" e impedisce il salvataggio.
			\end{altflow}
		\end{altflow}
	}
	\newpage
	\usecase
	{Modifica utente}
	{IF-2.2, UI-1.2.2}
	{Bibliotecario}
	{L’utente deve essere già registrato.}
	{Dati anagrafici aggiornati.}
	{
		\begin{enumerate}
			\item Il bibliotecario ricerca e seleziona l’utente da modificare.
			\item Seleziona "Modifica dati".
			\item Aggiorna i campi modificabili (Nome, Cognome, Email).
			\item Il bibliotecario conferma l’operazione.
			\item Il sistema salva le modifiche.
		\end{enumerate}
	}
	{
		\begin{altflow}[start=5]
			\item {Annullamento}
			\begin{altflow}
				\item L'operazione viene annullata senza salvare.
			\end{altflow}
		\end{altflow}
	}
    
	\usecase
	{Cancellazione utente}
	{IF-2.3, UI-1.2.2}
	{Bibliotecario}
	{L’utente è presente in archivio.}
	{L'utente viene rimosso dal sistema.}
	{
		\begin{enumerate}
			\item Il bibliotecario seleziona l’utente da eliminare.
			\item Seleziona il comando "Cancella utente".
			\item Il sistema verifica che l’utente non abbia prestiti attivi (libri non ancora restituiti).
			\item Il sistema chiede conferma.
			\item Il bibliotecario conferma.
			\item Il sistema elimina l'utente.
		\end{enumerate}
	}
	{
		\begin{altflow}[start=3]
			\item {Prestiti pendenti}
			\begin{altflow}
				\item Il sistema rileva prestiti attivi.
				\item Mostra errore: "Impossibile cancellare: l'utente ha libri in prestito".
			\end{altflow}
		\end{altflow}
	}
	
	\usecase
	{Visualizzazione utenti}
	{IF-2.4, UI-1.2.2}
	{Bibliotecario}
	{Database utenti inizializzato.}
	{Elenco visualizzato correttamente.}
	{
		\begin{enumerate}
			\item Il bibliotecario accede a "Area Utenti".
			\item Il sistema recupera la lista utenti.
			\item Il sistema mostra l'elenco ordinato per Cognome e Nome.
		\end{enumerate}
	}
	{
		\begin{altflow}[start=2]
			\item Nessun utente
			\begin{altflow}
				\item Il sistema mostra un avviso "Nessun utente registrato".
			\end{altflow}
		\end{altflow}
	}
	
	\usecasenohrule
	{Ricerca utente}
	{IF-2.5, UI-1.2.3}
	{Bibliotecario}
	{Criteri di ricerca noti (Cognome o Matricola).}
	{Lista filtrata degli utenti.}
	{
		\begin{enumerate}
			\item Il bibliotecario accede alla barra di ricerca in "Area Utenti".
			\item Inserisce la stringa di ricerca (Cognome o Matricola).
			\item Seleziona "Cerca".
			\item Il sistema mostra gli utenti che corrispondono ai criteri.
			\item Il bibliotecario seleziona l'utente desiderato.
		\end{enumerate}
	}
	{
		\begin{altflow}[start=4]
			\item Nessun risultato
			\begin{altflow}
				\item Il sistema mostra il messaggio "Utente non trovato".
			\end{altflow}
		\end{altflow}
	}
	
\usecase
    {Creazione nuovo prestito}
    {BF-1.1, DF-1.3, UI-1.3.2}
    {Bibliotecario}
    {Utente registrato e Libro esistente.}
    {Nuovo prestito attivo creato, copie disponibili decrementate.}
    {
        \begin{enumerate}
            \item Il bibliotecario accede a "Area Prestiti" e preme "Nuovo prestito".
            \item Seleziona l'utente richiedente dalla lista o tramite ricerca.
            \item Il sistema verifica che l'utente abbia meno di 3 prestiti attivi (BF-1.1).
            \item Il bibliotecario seleziona il libro da prestare.
            \item Il sistema verifica che le copie disponibili siano maggiori di 0.
            \item Il bibliotecario specifica la data prevista per la restituzione.
            \item Il bibliotecario conferma il prestito.
            \item Il sistema crea il record del prestito, registra la scadenza indicata e decrementa le copie disponibili del libro.
        \end{enumerate}
    }
    {
        \begin{altflow}[start=3]
            \item Limite prestiti raggiunto
            \begin{altflow}
                \item Il sistema segnala "Limite prestiti (3) raggiunto per questo utente".
                \item L'operazione viene bloccata.
            \end{altflow}
        \end{altflow}
        
        \begin{altflow}[start=5]
            \item Copie esaurite
            \begin{altflow}
                \item Il sistema segnala "Nessuna copia disponibile".
                \item L'operazione viene bloccata.
            \end{altflow}
        \end{altflow}
        
        \begin{altflow}[start=6]
            \item Data non valida
            \begin{altflow}
                \item Il sistema rileva una data antecedente a quella odierna.
                \item Il sistema richiede di inserire una data valida futura.
            \end{altflow}
        \end{altflow}
    }
\vspace{1cm}
	\usecase
	{Restituzione libro (Chiusura prestito)}
	{BF-1.2, UI-1.3.2}
	{Bibliotecario}
	{Esiste un prestito attivo per l'utente e il libro indicati.}
	{Prestito marcato come chiuso, copie disponibili incrementate.}
	{
		\begin{enumerate}
			\item Il bibliotecario accede a "Area Prestiti".
			\item Seleziona l'utente che sta restituendo il libro.
			\item Il sistema mostra la lista dei prestiti attivi per quell'utente.
			\item Il bibliotecario seleziona il prestito relativo al libro fisico riconsegnato.
			\item Preme il comando "Restituisci / Chiudi prestito".
			\item Il sistema aggiorna lo stato in "Restituito" e incrementa le copie disponibili del libro.
		\end{enumerate}
        
	}
	{
		\begin{altflow}[start=6]
			\item Scenari alternativi
			\begin{altflow}
				\item Se il prestito era scaduto, il sistema registra comunque la restituzione (eventuali sanzioni sono gestite esternamente al software in questa versione).
			\end{altflow}
		\end{altflow}
	}
	
	\usecase
	{Visualizzazione prestiti attivi}
	{BF-1.3, UI-1.3.1, UI-3.1}
	{Bibliotecario}
	{Nessuna precondizione particolare.}
	{Visualizzazione tabella monitoraggio prestiti.}
	{
		\begin{enumerate}
			\item Il bibliotecario accede alla dashboard "Area Prestiti".
			\item Il sistema mostra la tabella di tutti i prestiti in corso (non ancora restituiti).
			\item Il sistema evidenzia in rosso le righe corrispondenti ai prestiti con data di scadenza superata (UI-3.1) e in giallo le righe corrispondenti ai prestiti in fase di scadenza.
		\end{enumerate}
	}
	{
		\begin{altflow}[start=2]
			\item Nessun prestito
			\begin{altflow}
				\item Il sistema mostra "Nessun prestito in corso".
			\end{altflow}
		\end{altflow}
	}
    
		\usecase
{Ricerca prestito}
{IF-2.7}
{Bibliotecario}
{Esistono prestiti attivi a sistema.}
{Visualizzazione filtrata della lista prestiti.}
{
    \begin{enumerate}
        \item Il bibliotecario accede alla schermata "Area Prestiti".
        \item Inserisce un criterio di ricerca (Matricola utente, Cognome o Titolo libro).
        \item Seleziona il comando "Cerca".
        \item Il sistema filtra la lista mostrando solo i prestiti pertinenti.
    \end{enumerate}
}
{
    \begin{altflow}[start=4]
        \item \textbf{Nessun risultato}
        \begin{altflow}
            \item Il sistema comunica "Nessun prestito trovato per i criteri selezionati".
        \end{altflow}
    \end{altflow}
}

    \usecasenohrule
    {Cancellazione prestito (Annullamento errore)}
    {IF-2.8, UI-1.3.2}
    {Bibliotecario}
    {Esiste un prestito attivo (inserito per errore).}
    {Il record del prestito viene rimosso e le copie del libro incrementate.}
    {
        \begin{enumerate}
            \item Il bibliotecario individua il prestito errato nella lista.
            \item Seleziona il comando "Elimina/Annulla prestito".
            \item Il sistema chiede conferma dell'eliminazione definitiva.
            \item Il bibliotecario conferma.
            \item Il sistema rimuove logicamente il record del prestito e incrementa immediatamente le copie disponibili del libro.
        \end{enumerate}
    }
    {
        \begin{altflow}[start=4]
            \item \textbf{Annullamento operazione}
            \begin{altflow}
                \item Il bibliotecario nega la conferma.
                \item Il prestito rimane attivo e nessuna modifica viene applicata.
            \end{altflow}
        \end{altflow}
    }


   
\end{document}